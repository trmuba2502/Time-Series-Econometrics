\section{Question 3}
\textbf{i. What is the main research question of the paper? What are the three key variables studied? Which variables are endogenous and which are exogenous?}

\begin{itemize}
    \item The main research question is to measure the effects of monetary policy on the Vietnamese economy. Three key variables are interest rate, aggregate price, and output.
    \item Endogenous variables include Industrial production index, Consumer price index, Broad money, Central Bank policy rate, Total domestic credit, and Nominal exchange rate.
    \item Exdogenous variables are World oil price, and Chinese lending rate.
\end{itemize}

\textbf{ii. What are the stationarity properties of the variables? How did the authors handle seasonal adjustment of the data?}

\begin{itemize}
    \item Broad money, total credit, and the Chinese lending rates are stationary at level; the five other time series are stationary at first difference.
    \item The authors detected seasonality by regressing each variable on seasonal dummy variables and a yearly time trend. Additionally, the alternative solution, which uses a complete set of monthly dummy variables in the VAR models to deal with seasonality.
\end{itemize}

\[
Y_t = c_0 + qYear_t + \sum_{i=1}^{11} m_iD_{it} + \epsilon_t
\]

\textbf{iii. Which model is used in the study – VAR or SVAR? What method is employed to identify the non–white-noise residual matrix $u_t$?}
\begin{itemize}
    \item Model used in the study is VAR. Method employed is to identify the non-white-noiseresidual matrix $u_t$ is Cholesky decomposition of endogenous variables. The authors choose Cholesky decomposition because it imposes fewer restrictions.
\end{itemize}

\textbf{iv. Is this methodology purely atheoretical, or does it rely on theoretical a priori assumptions?}
It is not purely atheoretical, it based on theoretical framework and Granger-causality test. 
\begin{itemize}
    \item According to Christiano, Eichenbaum, and Evans (2005); Kim and Roubini (2000); Elbourne and de Haan (2009); Raghavan, Silvapulle, and Athanasopoulos (2012), the linkages between these variables: IPI $\rightarrow$ CPI \& M2 $\rightarrow$ Interest rate $\rightarrow$ Credit $\rightarrow$ Exchange rate.
    \item The Granger-causality test also shares the same results, \hl{IPI, CPI} Granger-cause \hl{M2, Inte, EXC}. M2 Granger-causes Inte and EXC at the 10\% significance level and Cred at the 1\% significance level. Inte Granger-causes Cred at 1\% significant level. EXC is Granger-caused by almost all variables (IPI, CPI, M2, Cred)
\end{itemize}

\textbf{v. Summarize the effect of a monetary policy shock on inflation and national output based on the impulse response functions.}

\hl{Response of National Output (Industrial Production)}: The analysis indicates that national output largely unresponse to monetary shocks in the short run, except money supply.
\begin{itemize}
    \item \textbf{Broad Money Supply} There is no impact during the first quarter (0–3 months), but a statistically significant positive effect from the 4th month onwards.
    \item \textbf{Other Variables} Shocks to the interest rate, exchange rate, and credit do not statistically significant.
\end{itemize}


\hl{Response of Price Level}: The price level sensitive to monetary instruments
\begin{itemize}
    \item \textbf{Interest rate} A shock to the interest rate has a significant negative effect on prices starting from the 3rd month and persisting until the 20th month. $\rightarrow$ monetary reduces inflation
    \item \textbf{Log of exchange rate} Shock to exchange rate do not statistically significant 
    \item \textbf{Log of broad money} The effect is statistically insignificant for the first two years (approx. 24 months), with a positive effect appearing only after the 26th month. $\rightarrow$ The money supply increase price level in long-term 
    \item \textbf{Log of credit} a shock to the credit level leads to a negative response in prices, which is statistically significant between the 2nd and 30th months.
\end{itemize}