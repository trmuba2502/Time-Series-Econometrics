\section{Question 1}
\textbf{i. Import the data, and limit the sample to the period of 1985-2019. Plot the three variables in a single graph}

\begin{figure}[H]
    \centering
    \includegraphics[scale=0.3]{Images/Question 1/line.png}
    \caption{Plot of ffr, inflation rate, and unemployment rate}
    \label{fig:q1-line}
\end{figure}

\textbf{ii. Estimate the VAR model to learn about the dynamic interrelationships among the three variables. Note: use varsoc to select the optimal lag lengths.}

\begin{figure}[H]
    \centering
    \includegraphics[scale=0.5]{Images/Question 1/varsoc.png}
    \caption{Find optimal lag}
    \label{fig:q1-varsoc}
\end{figure}

AIC smallest at \textbf{lag 5} $\rightarrow$ The optimal lag length is 5.

\[
\begin{aligned}
    inflation_t = \alpha_{10} + \sum_{i=1}^5\beta_{1i} inflation_{t-i} + \sum_{j=1}^5 \gamma_{1j} unrate_{{t-j}} + \sum_{k=1}^5 \theta_{1k} ffr_{t-k} + \epsilon_{1t} \\
    unrate_t = \alpha_{20} + \sum_{i=1}^5\beta_{2i} inflation_{t-i} + \sum_{j=1}^5 \gamma_{2j} unrate_{{t-j}} + \sum_{k=1}^5 \theta_{2k} ffr_{t-k} +\epsilon_{2t} \\
    ffr_t = \alpha_{30} + \sum_{i=1}^5\beta_{3i} inflation_{t-i} + \sum_{j=1}^5 \gamma_{3j} unrate_{{t-j}} + \sum_{k=1}^5 \theta_{3k} ffr_{t-k} +\epsilon_{3t}
\end{aligned}
\]

\hl{White noise test for residual}

\begin{figure}[H]
    \centering
    \begin{subfigure}[b]{0.495\textwidth}
        \centering
        \includegraphics[width=\textwidth]{Images/Question 1/res_ffr.png}
    \end{subfigure}
    \hfill 
    \begin{subfigure}[b]{0.495\textwidth}
        \centering
        \includegraphics[width=\textwidth]{Images/Question 1/res_infl.png}
    \end{subfigure}
    \hfill  
    \begin{subfigure}[b]{0.495\textwidth}
        \centering
        \includegraphics[width=\textwidth]{Images/Question 1/res_unrate.png}
    \end{subfigure}
    \caption{residual VAR(6)}
    \label{fig:residuals of VAR(6)}
\end{figure}

These graphs indicated that $\epsilon_t$ in 3 equations of VAR(5) is white noise

\begin{figure}[H]
    \centering
    \begin{subfigure}[b]{0.5\textwidth}
        \centering
        \includegraphics[width=\textwidth]{Images/Question 1/var_1.png}
    \end{subfigure}
    \hfill 
    \begin{subfigure}[b]{0.48\textwidth}
        \centering
        \includegraphics[width=\textwidth]{Images/Question 1/var_2.png}
    \end{subfigure}
    \hfill 
    \begin{subfigure}[b]{0.48\textwidth}
        \centering
        \includegraphics[width=\textwidth]{Images/Question 1/var_3.png}
    \end{subfigure}
    \hfill 
    \begin{subfigure}[b]{0.48\textwidth}
        \centering
        \includegraphics[width=\textwidth]{Images/Question 1/var_4.png}
    \end{subfigure}
    \caption{VAR(6)}
    \label{fig:VAR(6)}
\end{figure}

The estimated result shows:
\begin{itemize}
    \item lag 1, 2 of ffr, lag 4 of inflation, and lag 3 of urate affect \hl{ffr}
    \item lag 1, 4, 5 of inflation affect \hl{inflation}
    \item lag 1 of unrate affect \hl{unrate}
\end{itemize}

\textbf{iii. Perform the impulse response function (IRF) analysis and make a brief comments (4-6 lines) on it.}

\begin{figure}[H]
    \centering
    \includegraphics[scale=0.4]{Images/Question 1/irf.png}
    \caption{Impulse Respond Function}
    \label{fig:q1-irf}
\end{figure}

\hl{Response of Fed Fund Rate}
\begin{itemize}
    \item \textbf{To Own Shock (ffr)} The Federal Funds Rate response positively to its own lags
    \item \textbf{To Inflation Shock} The policy response is delayed. There is no effect in the early periods. however, the rate gradually have amoderate positive effect in the long run
    \item \textbf{To Unemployment Shock (unrate)} The rate shows no change in short-run but moderate negative response in the long run.
\end{itemize}
\hl{Response of Inflation rate}
\begin{itemize}
    \item \textbf{To Own Shock (inflation)} The response is positive during the early periods and then statistically insignificant after the 5th period.
    \item \textbf{To FFR Shock} The response of inflation to a monetary policy shock is not statistically significant.
    \item \textbf{To Unemployment Shock} The response of inflation to unemployment rate shock is not statistically significant.
\end{itemize}
\hl{Response of Unemployment Rate}
\begin{itemize}
    \item \textbf{To Own Shock (unrate)} The unemployment rate responds positively to its own shock.
    \item \textbf{To FFR Shock} There is no statistically significant response of unemployment to the shock of the Federal Funds Rate.
    \item 
    \item \textbf{To Inflation Shock} The response of unemployment to inflation shocks is approximately zero.
\end{itemize}

\textbf{iv. Perform the Granger causality test based on your VAR estimation and interpret the results.} 

\begin{figure}[H]
    \centering
    \includegraphics[scale=0.7]{Images/Question 1/granger.png}
    \caption{Granger Causality test}
    \label{fig:q1-granger}
\end{figure}

%Note: Granger test, H0: x (excluded) does not Granger-cause y (equation)
\hl{Interpretation}
\begin{itemize}
    \item We can reject the null hypothesis that unrate does not Granger-cause inflation $(p-value = 0.002) \rightarrow$ unrate Granger-cause inflation
    \item We can reject the null hypothesis that ffr does not Granger-cause unrate $(p-value = 0.001) \rightarrow$ ffr Granger-cause unrate
\end{itemize}