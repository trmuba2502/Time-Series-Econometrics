\section{LITERATURE REVIEW}
\label{sec: lit_review}
    \subsection{Theoretical Framework}
    \label{sec: theory}
        %theory
        Aligned with "free-trade", the problem of protecting the equality in income distribution among the countries has also been brought by \cite{StolperSamuelson1941}\footnote{Regarding the Heckscher-Ohlin model, the model assumes free trade between 2 countries with 2 goods and 2 factors of production. And Samuelson's study was based on that model foundation.}.
        The theorem proves that, in an unskilled labor-abundant country, trade increases demand for unskilled workers, thus raising their wages and reducing the income gap with skilled workers. In contrast, capital-abundant country exports their intensive goods, benefiting capital owners and high-skilled laborers, so that increasing inequality.
        Therefore, this study analyzes the validity of the Stolper and Samuelson theorem about the income disparity in the U.S. economy.
        
    \subsection{Previous Studies}
    \label{sec: pre_stu}
        The relationship between trade openness and income inequality has been a subject of intense debate, yielding mixed empirical results. A comprehensive meta-analysis by \cite{meta} indicates that while economic globalization generally has a small-to-moderate inequality-increasing impact, this effect is driven more strongly by financial globalization (e.g., FDI) than by trade globalization. Cross-country studies, such as \cite{meschi2009trade}, suggest that trade with high-income countries specifically worsens income distribution in middle-income countries due to technological upgrading, but the evidence for high-income economies remains complex. \\
        %meta: page 2978 Does economic globalisation affect income inequality? A meta-analysis. The World Economy
        %meschi2009trade: page 2 - summary

        Despite the scarcity of specify research papers in the US, there is some strong evidence showing how trade shocks directly affect inequality. A study by \cite{autor2013} examined the rise of Chinese imports between 1990 and 2007 reveals that local areas in the U.S. that faced more competition from China lost a significant number of manufacturing jobs and saw lower wages. This mainly hurt low-skilled workers, which directly led to a rise in inequality in those regions. % page 2159 THE AMERICAN ECONOMIC REVIEW October 2013 & abstract.
        Apart from potential dynamic linkage, \cite{barusman2017impact} uses OLS regression method to identify the influence of trade openness on income inequality in the U.S. via multiple independent variables. Regardless limit of the sample size (1970 -  2014), the estimation results when using Gini as dependent variables\footnote{The paper uses 2 types of dependent variables - Gini and top 10\% Income share} display that both 2 sides of trade (export and import) are associated with positive effect on income inequality, and the export side has a greater impact. \\
        %conclusion method barusman2017impact

        However, many studies argue that the impact of trade on inequality in the U.S. is relatively small. \cite{lawrence2008} counters the argument that trade drives inequality, positing that international trade accounts for only a small fraction of the rise in U.S. inequality. He suggests that the widening gap is primarily driven by skill-biased technological change rather than import competition. This perspective is supported by \cite{krugman2008}, who revisited the trade-wage debate and concluded that the effect of trade on aggregate inequality remains modest. Krugman argued that although imports from developing countries have increased, the volume of trade is still insufficient to be the primary culprit for the massive surge in the U.S. wage gap.
          
        Besides, the role of fiscal policy, specifically taxation and government spending, remains central to modern distributional analysis. \cite{rojas2025} provides strong empirical evidence from the 1990 to 2019 period demonstrating that economic growth alone is insufficient to reduce inequality unless accompanied by "prudent fiscal policies" in some cases. In Brazil, progressive fiscal interventions and targeted social policies were instrumental in reducing the Gini index from over 60 in 1990 to 53.5 by 2019, despite market pressures. Additionally, their analysis of 'Shared Socioeconomic Pathways' (SSPs) suggests that in the future, inequality will be determined largely by the inclusiveness of social policies rather than market forces. Consequently, omitting these fiscal variables from an analysis of trade and inequality could lead to significant omitted variable bias. \\    
        % page 16 Global inequality and economic growth: The Three Decades before Covid-19 and Three Decades After.

        Other research about inequality discloses that AI and automation are currently the important determinant of labor demand \cite{acemoglu2024}. He argues that these technologies are displacing routine tasks performed by low and middle skilled workers, and complementing high skilled labor, thereby worsen the wage gap more significant than trade exposure
        % abstract The Simple Macroeconomics of AI
    \subsection{Methodological Approaches}
    \label{sec: Method_App}
        In time series data analysis, researchers typically choose between univariate models and multivariate frameworks. Univariate models, such as AutoRegressive Integrated Moving Average (ARIMA), are widely used for forecasting a single variable based on its own past values. However, the limitation of ARIMA is that it can only analyze isolate variable and its own lags. For this study, using a univariate approach would be insufficient because it cannot capture the dynamic relationship between other potential variables and income inequality.\\

        In order to address this limitation, this study utilizes the Vector Autoregression (VAR) framework proposed by \cite{sims1990inference}. The VAR model is superior for this research because it is a multivariate system that treats all variables as endogenous. This allows us to examine the bidirectional feedback loops between trade and inequality. Furthermore, to analyze the transmission mechanism of economic shocks, we utilize Impulse Response Functions (IRFs) derived from the VAR system. As proved by \cite{stock2001}, IRFs are the primary tool for structural analysis, allowing researchers to isolate the effect of a specific shock to one variable on the time path of another variable while holding other shocks constant.
        %page 106, "Impulse Responses"