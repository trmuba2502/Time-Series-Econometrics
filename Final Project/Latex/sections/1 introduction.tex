\section{INTRODUCTION}
\label{sec:introduction}
    %motivation
    Throughout the history of the economy, revolution has gone hand in hand with globalization. Since the existence of the Silk Road, trade had stopped being a local or regional affair and started to become global  \citep{wefglobalization}. Then, economists have studied the benefits of specialization, trade openness, win-lose relationship of collaboration between specialized countries \citep{ricardo1817principles}. In addition, the break in the global chain due to COVID-19, or Donald Trump's ideology of MAGA\footnote{Making America Great Again}, once again raised the world's concern about "globalization". \\

    %Research gap
    Although theoretical models like Stolper-Samuelson states that trade openness widens the income gap in developed economies like the U.S., empirical evidence remains mixed. While some researchers argue trade is the main driver of income inequality, others point to technological change. Existing literature presents two significant gaps that this study aims to address. First, most studies focus on cross-country analysis. There is a scarcity of studies specifically examining the U.S. economy over a long period to capture the trade-inequality relationship. Second, and most importantly, many previous studies overlook the role of fiscal policy. Taxation and spending, for instance, via which governments commonly redistribute income and mitigate inequality. Therefore, ignoring these variables can lead to bias, making the estimated impact of trade inaccurate.\\

    %Research objective
    This paper aims to examine the relationship between trade openness and income inequality in the U.S. economy. Specifically, we seek to determine whether a significant linkage exists and if the interaction between trade and inequality follows an error correction mechanism. Furthermore, this study distinguishes itself by incorporating fiscal policy variables (taxation and government spending) to control for omitted variable bias.