\section{IMPULSE RESPONSE FUNCTION}
\label{sec: IRF} 
\subsection{Impulse response function}
\label{subsec: Impulse response function}
    The objective of the dynamic relationships between variables in this paper is exhibited via computing the impulse response function (IRFs). \Cref{fig:irf_trade} shows the reaction of the Gini index, Government Spending, Tax revenues, and Trade openness to a 1\% shock in Trade openness, which is the main objective; other IRFs can be found in \Cref{fig:irf_tax}, \Cref{fig:irf_spend}, \Cref{fig:irf_gini}. The result suggests that with a 1\% increase in Trade openness, the Gini index has no significant reaction, as same for Tax revenues. Whereas, Government spending has a significant instant reaction, which means in the short term (i.e., lag 0-2), Government spending dropped and gradually increased back to 0. In the long term (from lag 6-8), the spending variable only slightly increases. 

    \begin{figure}[H]
        \centering       
        \caption{IRF to a 1\% of trade shock}
        \includegraphics[scale=0.3]{images/IRF_trade.png}
        \label{fig:irf_trade}
    \end{figure}
    \Cref{tab:irf_trade} illustrates the numerical reaction of variables to the shock of Trade openness.
    \begin{table}[H]
    \centering
    \caption{The response to a 1\% Trade Shock}
    \label{tab:irf_trade}
    \begin{threeparttable}
        \footnotesize 
        \begin{tabular}{l c c c c c c c} 
            \toprule
            \textbf{Variables} & \textbf{Impact} & \textbf{1y} & \textbf{2y} & \textbf{3y} & \textbf{4y} & \textbf{5y} & \textbf{6y} \\ 
            \midrule
            Gini index & 0.000 & 0.082 & -0.086 & -0.062 & 0.062 & 0.097 & 0.095 \\ \addlinespace[0.07cm]
            Trade openness & \textbf{1.097} & \textbf{0.656} & 0.281 & \textbf{0.463} & \textbf{0.577} & \textbf{0.525} & 0.422 \\ \addlinespace[0.07cm]
            Tax revenues & 0.123 & -0.029 & \textbf{-0.236} & -0.101 & -0.036 & 0.066 & 0.042 \\ \addlinespace[0.07cm]
            Government spending & \textbf{-0.638} & -0.326 & 0.151 & 0.236 & 0.230 & -0.008 & -0.130 \\
            \bottomrule
        \end{tabular}
        \begin{tablenotes}[flushleft]
            \footnotesize
            \item \textit{Notes:} Bold numbers indicate significance at the 5\% confidence level.
        \end{tablenotes}
    \end{threeparttable}
\end{table}
    
    The null-hypothesis of this study is the significant dynamic response of the Gini index to a 1\% shock in Trade openness; however, the result rejects the hypothesis. Instead, IRFs of Trade openness shocks predict a significant linkage between Trade openness and Government spending. That negative immediate response of Government spending (-0.638 at 5\% significance level) indicates the development of the economy, driven by a rise in trade volume, leads to a decrease in Government spending, which provides evidence for the counter-cyclical fiscal policy in the U.S.
\subsection{Discussion}
\label{subsec: Discussion}
    As shown in \Cref{fig:lines}, Gini exhibits a persistent trend, while Government Spending has a cyclical fiscal stance. These movements may imply the property of consistency in sociology indicators  (i.e., Gini) with a high tendency and the sensitivity of economic factors to the shocks of an economic situation. Although the insignificant response of the Gini index contradicts the theory of \cite{StolperSamuelson1941}, this finding aligns with previous U.S. studies, such as \cite{lawrence2008} and \cite{krugman2008}. These authors argue that trade plays a minor role in rising inequality compared to technological changes, as trade volumes are insufficient to drive the aggregate wage gap. Consequently, our results reinforce the perspective that trade openness is not the primary determinant of income inequality in the U.S.. \\
    
    The review by \cite{harrison2011recent} summarizes some potential reasons for the inconsistency of empirical work with the HO framework\footnote{Stolper-Samuelson model is one of the implications of Heckscher-Ohlin model}, and that simple approach is a guide to the trade and inequality problem, but requires more qualifications when it comes to the real world.    
    %Page 4 &  17 harrison2011recent 
    In the empirical work part of the summary paper \citep{harrison2011recent}, authors also highlight the incomplete understanding when ignoring the assumptions of the framework, such as perfect competition, free trade, and factor mobility\footnote{Not allow for "immigration" of factors of production (i.e., capital and labour)}.
    %Page 17 harrison2011recent 
    Consequently, the classical framework does not account for the non-tradable goods and services, and the offshoring situation. As demonstrated by \cite{feenstra1999impact}, the reallocation of task-abundance responsibilities between involved countries increases wages in both countries\footnote{"Offshoring increases the relative demand for skilled labor in both countries involved because the offshored tasks are more skill intensive than those previously performed in the country to which they were offshored, but they are less skill intensive than those in the country that is doing the offshoring"\citep{harrison2011recent}}. \\
    %thầy hỏi thì: kí kết NAFTA, WHO  -> các big firm chuyển sang sản xuất ở các nước có cheap labour ~ offshoring 
    
    Our findings offer a nuanced perspective when compared to the influential work of \cite{autor2013}. Their study, focusing on local labor markets specifically exposed to the 'China Shock,' found that import competition significantly harmed low-skilled labor and exacerbated inequality in those specific manufacturing-heavy regions. In contrast, our study employs a macro-level VAR approach to examine the aggregate dynamics of the entire U.S. economy. \\

    The fact that our study finds no statistically significant response of the national Gini coefficient to trade shocks suggests that while trade may produce concentrated losses in specific sectors or regions (as identified by \cite{autor2013}), these effects do not necessarily translate into a deterioration of overall national income distribution. This distinction provides a positive signal for policymakers. It implies that the classic trade-off between economic openness and equity may be less severe at the aggregate level than previously feared. Consequently, policymakers can pursue trade liberalization to foster growth, provided that they implement targeted safety nets for specific affected areas, without being paralyzed by the dilemma that open trade inevitably worsens nationwide inequality.
