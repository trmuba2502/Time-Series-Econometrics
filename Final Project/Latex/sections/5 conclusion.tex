\section{CONCLUSION}
\label{sec:conclusion}
    \subsection{Summary}
    \label{sec: Summary}
    This study empirically examined the dynamic relationship between trade openness and income inequality in the United States over the period from 1972 to 2023. To ensure the unbiasedness, this paper incorporated fiscal policy variables (tax revenue and government spending) into a Vector Autoregression (VAR) framework.\\

    The empirical results derived from the Impulse Response Functions (IRFs) lead to two main conclusions. First, contrary to the predictions of the Stolper-Samuelson theorem, we found no statistically significant response of the Gini index to shocks in trade openness. This finding aligns with the "technology-driven inequality" hypothesis supported by \cite{lawrence2008} and \cite{krugman2008}. Second, the study reveals a significant interaction between trade and fiscal policy. Specifically, a positive shock to trade openness leads to an immediate and significant decline in government spending. This provides evidence of counter-cyclical fiscal behavior (or automatic stabilizers), implying that trade expansion often coincides with economic growth periods, thereby reducing the government deficit.\\

    In conclusion, our findings suggest that while trade openness plays a crucial role in influencing U.S. fiscal dynamics, it is not the primary driver of the rising aggregate income inequality observed over the past five decades.

    \subsection{Limitations}
    \label{sec: Limitations}
    %t search thì limitations hay ở cuối discussion á m....
    Despite these findings, this study has some limitations that need to be taken into consideration. Firstly, the observed time period is from 1972 to 2023 (51 years), though the data is enough to conduct testing, the limited size of the data sample may be inefficient to draw conclusions for the whole population. Secondly, since Gini is a surveyed index, it may have bias in the way of conducted (including the answers from the respondents). In addition, Gini is an indicator that represents income equality; therefore, it cannot reflect all equality features, gender equality, for instance. Furthermore, Trade openness variable data in this study is just the monetary value of trading goods and services; thus, it does not separate the de facto and de jure \citep{Grabner2020}\footnote{In brief, de facto stands for the actual flows of trade, including financial flows. De jure measures trade openness by evaluating trade-related policies/restrictions} measurements among trade openness levels. Lastly, the Vector Autoregressive (VAR) model is sensitive to the lag length selection; though we have clarified the selection, there is still potential effect that may lead to incorrect in IRFs results.      
    %de jure 89  
    